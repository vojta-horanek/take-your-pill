\documentclass[../TakeYourPill.tex]{subfiles}
\graphicspath{{\subfix{images/}}}

\begin{document}

Obrazovka léku (\texttt{DetailsFragment}) je vytořena jako fragment s rozložením \texttt{fragment\_details.xml}. Na obrazovce lze vidět název léku, jeho popis, fotografii, připomínky, příjem a poslední odeslanou připomínku. Pokud nějakou z položek lék neobsahuje (například fotografii), místo pro ni se skryje. Při kliknutí na fotografii se fotografie zobrazí v plné velikosti. Pokud obrazovku otevřeme z oznámení, nebo má lék nepotvrzenou připomínku v posledních 30 minutách, zobrazí se nad názvem léku karta vyznívající k potvrzení této připomínky. Na spodní části obrazovky jsou tlačítka \enquote{smazat}, \enquote{historie} a \enquote{upravit}. Tlačítko \enquote{smazat} otevře dialog, kde uživatel může zvolit, zda chce smazat pouze lék a zachovat jeho historii, nebo ho smazat i s historií. Dialog je implementovaný ve třídě \texttt{DeleteDialog}. Tlačítko \enquote{historie} otevře dialog, ve kterém se zobrazí historie pro tento lék (\texttt{HistoryViewDialog}). Více o tomto dialogu naleznete v kapitole \ref{sec:historydialog}. Tlačítko \enquote{upravit} otevře obrazovku \texttt{EditFragment}, kde může uživatel lék upravit. Více o této obrazovce v následující sekci.

\end{document}
