\documentclass[../TakeYourPill.tex]{subfiles}
\graphicspath{{\subfix{images/}}}

\begin{document}

Obrazovka léku (\path{DetailsFragment}) je implementována jako fragment s layoutem \path{fragment_details.xml}. Na obrazovce lze vidět název léku, jeho popis a fotografii (pokud tyto položky má), připomínky a příjem. Při dlouhém podržení na fotografii se fotografie zobrazí v plné velikosti. Pokud obrazovku otevřeme z oznámení, nebo má lék nepotvrzenou připomínku v posledních 30 minutách, zobrazí se nad titulkem karta vyznívající k potvrzení této připomínky. Na spodní části obrazovky jsou tlačítka \enquote{smazat}, \enquote{historie} a \enquote{upravit}. Tlačítko \enquote{smazat} otevře dialog, kde uživatel může zvolit, zda chce smazat pouze lék a zachovat jeho historii, nebo ho smazat i s historií. Dialog je implementovaný ve třídě \path{DeleteDialog}. Tlačítko \enquote{historie} otevře dialog, ve kterém se zobrazí historie pro tento lék. Více o tomto dialogu naleznete v sekci \enquote{\nameref{sec:overview}} na straně \pageref{sec:overview}. Tlačítko \enquote{upravit} otevře \path{EditFragment}, kde může uživatel lék upravit. Více o této obrazovce v následující sekci.

\end{document}
