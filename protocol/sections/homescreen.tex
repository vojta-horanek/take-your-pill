\documentclass[../TakeYourPill.tex]{subfiles}
\graphicspath{{\subfix{images/}}}

\begin{document}

Domovská obrazovka, neboli sekce \enquote{Léky} (\texttt{HomeFragment}), je fragment obsahující pouze \texttt{RecyclerView} a \texttt{ExtendedFloatingActionButton} (dále jen FAB). FAB se při posunutí seznamu v RecyclerView zmenší, zvětší se až když seznam posuneme na začátek. 

Pro zobrazení dat v RecyclerView je potřeba mít \texttt{RecyclerAdapter}. Tato třída se stará o zobrazování dat s příslušným \texttt{ViewHolderem}. Adaptér, který je nastavený na tomto RecyclerView se jmenuje \texttt{AppRecyclerAdapter}. Každý \texttt{ViewHolder} pro objekty třídy \texttt{Pill} obsahuje kartu, jejíž rozložení je definováno v souboru \texttt{item\_pill.xml}. Na kartě se zobrazují všechny potřebné informace o léku: název, popis, barva, fotografie, připomínky a příjem. Pokud uživatel nepotvrdil nejnovější připomínku v posledních 30 minutách, zobrazí se na kartě i výzva k potvrzení. Po kliknutí na kartu se otevře detail příslušného léku. Snímek obrazovky sekce \enquote{Léky} lze vidět na straně \pageref{fig:homescreen} (obrázek \ref{fig:homescreen}).

\boxik{
\textbf{ViewHolder} (v doslovném překladu \enquote{držitel pohledu}) je třída, která se stará o zobrazení jedné položky v RecyclerView adaptéru. Má za úkol nastavit rozložení tak, aby odpovídalo vstupním datům (např. jednomu léku). Pro každý typ položky, který chceme zobrazovat si musíme tuto třídu musíme sami definovat.
}


\boxik{
\textbf{AppRecyclerAdapter} je \texttt{RecyclerAdapter} založený na \texttt{ListAdapter}. Tento adaptér se používá pro většinu seznamů v aplikaci, jelikož umožnujě přidat titulek a zobrazit prázdný stav. Toto je dosaženo přepsáním metody \texttt{submitList} a dosazením speciálních položek (\texttt{HeaderItem} a \texttt{EmptyItem}). Adaptér podporuje třídy, které dědí z \texttt{BaseModel}, jmenovitě \texttt{Pill}, \texttt{HistoryPillItem}, \texttt{HeaderItem} a \texttt{EmptyItem}.
}


\end{document}
