\documentclass[../TakeYourPill.tex]{subfiles}
\graphicspath{{\subfix{images/}}}

\begin{document}

Druhou sekcí historie je fragment \path{HistoryChartFragment}, který se stará o zobrazení grafů. Fragment obsahuje kartu se třemi koláčovými grafy, touto kartou lze posouvat pro zobrazení i těch grafů, které se na obrazovku nevešly. Pro grafy jsem využil knihovnu \textit{MPAndroidChart} \cite{chart} dovolující vytvoření velkého množství typů grafů. Já jsem použil jen graf koláčový, protože ostatní grafy by v této aplikaci nedávaly smysl. Všechny grafy v aplikaci jsou procentuální a knihovna si procenta vypočítá sama, aplikace počítá pouze počet položek, nikoliv jejich procentuální množství. Pokud se v grafu zobrazují jednotlivé léky (první a druhý graf), barva výseče souhlasí s barvou léku. 

\paragraph{Graf 1} Prvním grafem je graf s názvem \enquote{všechny připomínky}. Graf zobrazuje procentuální poměr připomínek jednotlivých léků. Data se získávají následujícím způsobem:

\begin{lstlisting}[language=Kotlin]
val allEntries = ArrayList<PieEntry>()
val pillsHistory = history.groupBy { it.pillId }.values
pillsHistory.forEach { pillHistory ->
    val pill = getPill(pillHistory.first().pillId)
    allEntries.add(PieEntry(pillHistory.size.toFloat(), pill.name))
}
\end{lstlisting}

Tento kód vypočítá délku každé skupinu historie indikovanou podle ID léku a přidá jí do seznamu používaném pro vykreslení grafu.


\paragraph{Graf 2} Druhý graf na obrazovce pod názvem \enquote{vynechané připomínky} zobrazuje procentuální poměr vynechaných připomínek pro jednotlivé léky. Pokud neexistují žádné vynechané připomínky, graf se nezobrazí. Data pro tento graf se získávají následovně (kód upraven pro snazší orientaci):

\begin{lstlisting}[language=Kotlin]
val missedEntries = ArrayList<PieEntry>()
val pillsHistory = history.groupBy { it.pillId }.values
pillsHistory.forEach { pillHistory ->
    val pill = getPill(pillHistory.first().pillId)
    val pillMissed = pillHistory.count { !it.hasBeenConfirmed }
    if (pillMissed > 0) {
        missedEntries.add(PieEntry(pillMissed, pill.name))
    }
}
\end{lstlisting}

Tento kód vypočítá počet nepotvrzených připomínek každé skupinu historie indikovanou podle ID léku a pokud je větší než nula, přidá jej do seznamu používaném pro vykreslení grafu.


\paragraph{Graf 3} Poslední graf s názvem \enquote{potvrzené x vynechané} zobrazuje procentuální poměr potvrzených a vynechaných připomínek. Data pro poslední graf se získávají tímto způsobem (kód upraven pro snazší orientaci):

\begin{lstlisting}[language=Kotlin]
val totalConfirmed = history.count { it.hasBeenConfirmed }
val totalMissed = history.size - totalConfirmed
val confirmedEntries: ArrayList<PieEntry> = arrayListOf(
        PieEntry(totalConfirmed, R.string.confirmed),
        PieEntry(totalMissed, R.string.missed)
)
\end{lstlisting}

Výpočet pro tento graf je nejjednodušší, pouze se vypočítá počet potvrzených a počet vynechaných (od velikosti veškeré historie odečtené potvrzené) a tyto hodnoty se přidají do seznamu používaném pro vykreslení grafu.

I když jsem zde v textu kód rozdělil do tří sekcí, v aplikaci je kód na jednom místě (\path{HistoryChartViewModel}) a využívá pouze jeden cyklus, je tak efektivnější a méně se dotazuje databáze.

\end{document}
