\documentclass[../TakeYourPill.tex]{subfiles}
\graphicspath{{\subfix{images/}}}

\begin{document}

Připomínání je provedeno standardními oznámeními v systému Android. Pomocí třídy \texttt{NotificationCompat} a \texttt{NotificationManagerCompat} jsem vytvořil oznámení s nejvyšší možnou prioritou, které obsahuje název léku, popis, říkající kdy si má uživatel lék vzít, a případnou fotografii. Oznámení má také barvu shodnou s barvou léku. Pod tímto obsahem jsou tlačítka \enquote{Potvrdit} a \enquote{Odložit}. Tlačítka mají každé přiřazené svůj \texttt{PendingIntent}, který pošle tzv. \textit{broadcast} (vysvětleno níže). Při kliknutí na oznámení se uživatel dostane do aplikace, kde je otevřený lék, kterému patří daná připomínka.

Postup \enquote{algoritmu} na připomínání je následující:

\begin{enumerate}
  \item Při ukládání léku, spuštění operačního systému, nebo spuštění aplikace se zavolá funkce \texttt{ReminderManager.planNextPillReminder()}, která plánuje připomínky pro jeden lék.
  \item V této funkci:
  \begin{itemize}
    \item Seřadíme připomínky léku podle času vzestupně.
    \item Procházíme jednotlivé připomínky.
    \item Pokud narazíme na připomínku, která má dnes teprve nastat, naplánujeme ji a funkci ukončíme.
    \item Pokud nenarazíme na žádnou připomínku, která nastane ještě dnes, naplánujeme první připomínku zítra a funkci ukončíme.
    \item Jinak funkci ukončíme a neplánujeme nic.
  \end{itemize}
  \item Pro naplánování používáme funkci \texttt{ReminderManager.createAlarm()}. Funkce použije systémovou službu \texttt{AlarmManager} a naplánuje s ní \enquote{alarm}, který spustí \texttt{ReminderReceiver} v čase dané připomínky.
  \item V této třídě:
  \begin{itemize}
    \item Pokud to není první připomenutí léku a je v daný den první, posuneme jeho iteraci cyklu.
    \item Zkontrolujeme, zda je lék aktivní (kvůli omezení na počet dní, nebo kvůli cyklu).
    \item Pokud je lék aktivní přidáme do historie toto připomenutí, naplánujeme \texttt{CheckReminder} a zobrazíme oznámení.
    \item Pokud lék aktivní není a zároveň je omezený na počet dní (tj. už nebude připomínat), odejdeme z přijímače.
    \item Jinak naplánujeme další připomínku a z přijímače odejdeme.
  \end{itemize}
\end{enumerate}

%TODO Reorder this

Broadcasty jsou důležitou součástí celého systému připomínání. V aplikaci existuje celkem 5 tříd, které broadcasty přijímají. Pro rozesílání broadcastů se používá objekt třídy \texttt{PendingIntent}, který v sobě navíc obsahuje \texttt{Intent}. Do intentu je možné vložit vlastní hodnoty, které lze v přijímačích broadcastů získat. Tímto způsobem lze například poslat ID připomínky. Následuje přehled tříd, které jsou v aplikaci používány na příjímání broadcastů:

\begin{table}[h!]
\centering
  \begin{tabular}{ |p{0.25\linewidth} | p{0.70\linewidth}| }
    \hline
    \texttt{BootReceiver} & Speciální přijímač, který se zavolá při spuštění operačního systému. Stará se o naplánování připomínek ke všem lékům.\\
    \hline
    \texttt{CheckReceiver} & Přijímač, který plánuje a zobrazuje oznámení, pokud uživatel připomínku nepotvrdil. \\
    \hline
    \texttt{ConfirmReceiver} & Přijímač volaný při stisknutí tlačítka \textit{Přijmout} v oznámení, potvrdí přijmutí léku v danou chvíli. \\
    \hline
    \texttt{DelayReceiver} & Přijímač volaný při stisknutí tlačítka \textit{Odložit} v oznámení, odloží oznámení o nastavený počet minut (ve výchozím stavu 30 minut). \\
    \hline
    \texttt{ReminderReceiver} & Nejdůležitější přijímač, který je volán systémem android v naplánovaný čas. Stará se o zobrazení oznámení, naplánování přijímače \texttt{CheckReceiver} a naplánování sama sebe na další připomínku u léku. \\
    \hline
  \end{tabular}
\end{table}

\end{document}
