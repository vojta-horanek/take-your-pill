\documentclass[../TakeYourPill.tex]{subfiles}
\graphicspath{{\subfix{images/}}}

\begin{document}

Druhá hlavní sekce aplikace se nazývá \enquote{historie}. Uživatel zde může pozorovat, kdy mu přišly připomínky, kdy a jestli si lék vzal a jaké množství si vzal (nebo s vzít měl). V této sekci jsou dostupné i tři koláčové grafy, které uživateli vizuálně ukazují jeho statistiky. Sekce historie je vytovřena jako fragment, který je prvky \texttt{ViewPager2} a \texttt{TabLayout} rozdělen na dvě části. Tento fragment (\texttt{HistoryFragment}) neobsahuje žádnou logiku ohledně historie, pouze se stará o nastavení výše zmíněných prvků.

\boxik{
\textbf{ViewPager2} a \textbf{TabLayout} jsou prvky, které rozdělují obrazovku na dva nebo více panelů, které jsou přepínatelné pomocí posouvání. \texttt{ViewPager2} je nová verze \texttt{ViewPager}, která interně využívá \texttt{RecyclerView}, jehož položkami jsou fragmenty. \texttt{TabLayout} pouze zobrazuje záložky nad panely a dovoluje jejich přepínání pomocí stisku.
}

\end{document}
