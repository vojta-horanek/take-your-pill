\documentclass[../TakeYourPill.tex]{subfiles}
\graphicspath{{\subfix{images/}}}

\begin{document}

Obrazovka \enquote{Nový lék} (\texttt{EditFragment}) má dvě funkce. Slouží jako obrazovka pro přidávání nového léku, a zároveň se používá pro úpravu léku. Titulek obrazovky se mění dle použití. Opět je implementována jako fragment. Pro každý lék je možno nastavit název a popis. Tyto dvě hodnoty se zapisují do prvku \texttt{TextInputLayout} z knihovny \textit{material} \cite{materialdesign-android}. Následuje nastavení barvy. Uživatel má na výběr ze sedmi barev: modrá, tmavě modrá, tyrkysová, zelená, žlutá, oranžová a červená. Vybírání barvy je implementováno pomocí prvku \texttt{RecyclerView} s atributem \texttt{orientation} nastaveným na hodnotu \texttt{horizontal}, zobrazí se tedy jako horizontální seznam. Jednotlivé barvy jsou definované třídou \texttt{PillColor}. Dále si uživatel nastaví připomínky. Při kliknutí na připomínku nebo na tlačítko \enquote{přidat připomínku} se zobrazí \texttt{ReminderDialog} kde uživatel může upravit/vytvořit připomínku. U připomínky lze nastavit i množství léku, jaké si v daný čas má uživatel vzít. Pro jeden lék nelze nastavit dvě připomínky se stejným časem, každá připomínka musí mít unikátní čas. 

Důležitým krokem je léku nastavit příjem. Uživatel si může zvolit, zda lék bere neustále, jen určitý počet dní, nebo v cyklu X dní aktivních, Y dní neaktivních. Tento prvek\footnote{prvkem je myšlen prvek uživatelského rozhraní, neboli \texttt{view}} je implementován v \texttt{PillOptionsView}, který dědí z \texttt{LinearLayout} a používá layout \texttt{layout\_pill\_options\_view.xml}. Veškerá logika výběru příjmu je implementována v této třídě. Jediné, o co se \texttt{EditFragment} musí postarat, je získání \texttt{ReminderOptions} (vysvětleno v kapitole \ref{sec:reminder}) z tohoto prvku pomocí metody \texttt{getOptions()}.

K léku lze přidat i fotografii. Prvek na výběr fotografie je implementován v \texttt{ImageChooserView}. Tato třída dědí z \texttt{LinearLayout} a používá layout definovaný v \texttt{layout\_image\_chooser.xml}. Pokud lék již nějakou fotografii obsahuje, prvek zobrazí tlačítko na její odstranění. Při výběru fotografie je použita knihovna \textit{EasyPermissions} \cite{easy-permissions} pro zajištění potřebných oprávnění a upravená verze knihovny \textit{imagepicker} \cite{imagepicker}. Knihovnu jsem upravil tak, aby respektovala vzhled aplikace. Zaprvé již nepoužívá standardní \texttt{AlertDialog}, nýbrž \texttt{BottomSheetDialog} a také vzhled tohoto dialogu byl upraven, aby odpovídal všem ostatním dialogům v aplikaci. Uživatel si může zvolit, zda chce fotografii vybrat z galerie, nebo chce vyfotit fotografii novou. Po vybrání/vyfocení se uživateli ukáže obrazovka, kde může fotografii upravit. Pro úpravu fotografie jsem použil knihovnu \textit{uCrop} \cite{ucrop}, kterou jsem také upravil. Oproti originální verzi se liší v použití prvků na navigaci, ikonách, podporou automatického tmavého vzhledu a sladěním do vzhledu aplikace. Knihovna uživateli umožní fotografii oříznout, otočit a škálovat. Knihovna má dvě verze: jednu nativní, napsaná v C++, a druhou standardní. I když jsem z počátku zvolil verzi nativní, v konečné verzi aplikace jsem použil standardní verzi knihovny. Důvodů jsem měl hned několik; V nativní verzi nebyly podporované určité formáty fotografií (jmenovitě \textit{.HEIC}), její plynulost nebyla v porovnání se standardní verzí znatelná, a navíc přidala k velikosti aplikace cca 1.5 MB.

Po nastavení všech hodnot může uživatel lék uložit pomocí \texttt{FAB} tlačítka. Aplikace lék vloží do databáze (pokud již existuje, tak jej aktualizuje) a naplánuje jeho následující připomínku.

\boxik{
\textbf{PillColor} je třída, používaná pro ukládání barvy léku. Obsahuje atributy \texttt{resource} a \texttt{isChecked}. Atribut \texttt{resource} ukládá \texttt{id} barvy uložené v \textit{App resources} \cite{resource}. Atribut \texttt{isChecked} vyjadřuje, zda je barva vybraná. Tento atribut se používá k zobrazování seznamu barev a zjištění, jakou barvu uživatel vybral.
}


\end{document}
