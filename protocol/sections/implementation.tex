\documentclass[../TakeYourPill.tex]{subfiles}
\graphicspath{{\subfix{images/}}}

\begin{document}

Při vytváření aplikace jsem vycházel ze zadání a využil jsem vlastních znalostní a zkušeností. Na naprogramování aplikace jsem použil programovací jazyk Kotlin. Zvolil jsem ho proto, že je preferovaný společností Google a oproti jazyku Java má mnoho výhod. Mnoho Android knihoven vychází právě pro Kotlin a tak mi jeho použití ulehčilo mnoho práce při programování. Jmenovitě knihovny z rodiny Android Jetpack \cite{jetpack} jsem použil hojně.

Uživatelské rozhraní aplikace je napsáno v jazyce XML. Pro jeho manipulaci jsem použil knihovny \textit{ViewBinding} \cite{viewbinding} a \textit{DataBinding} \cite{databinding}.

Při samotném vývoji jsem používal vývojové prostředí \textit{Android Studio} \cite{studio} a emulátor \textit{Android Emulator}.

Aplikace je napsána tak, aby odpovídala architektuře \textbf{Model-View-ViewModel}. Znamená to, že každá obrazovka má svůj \textit{ViewModel} a každá datová sekce má svůj \textit{Repozitář}. Tyto třídy jsou odděleny od samotných fragmentů a aktivit.


\boxik{
\textbf{ViewModel} je třída obsahující data a funkce pro její fragment/aktivitu, která má vlastní životní cyklus. Díky ViewModelu data přežijí změnu konfigurace jako například otočení obrazovky.
}


\boxik{
\textbf{Repozitář} (repository) je třída, která shromažďuje data z různých zdrojů a nabízí je ve vhodné formě ostatním třídám (například může data uchovat v mezipaměti). V této aplikaci přistupuje repozitář pouze do databáze a ve většině případů přímo volá funkce implementované v databázové vrstvě.
}

\end{document}
